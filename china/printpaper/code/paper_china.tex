\documentclass[11pt]{article}
\usepackage{amsmath,amssymb,amsfonts,amsthm}
\usepackage{verbatim}
\usepackage{caption}
\usepackage{graphicx}
\usepackage{natbib}
\usepackage{bibentry}
\usepackage{url}
\usepackage{wrapfig}
\usepackage[margin=1in]{geometry}
\usepackage{framed}
\usepackage{placeins}
\usepackage{bbm}
\usepackage{enumerate}
\usepackage{booktabs}
\usepackage{subfig}   % Supports \subfloat within \figure{} environment
\usepackage{multirow}
\usepackage[setpagesize=false,breaklinks=true]{hyperref}
\hypersetup{
    colorlinks=true,       % false: boxed links; true: colored links
    linkcolor=red,          % color of internal links
    citecolor=black,        % color of links to bibliography
    urlcolor=blue           % color of external links
}

\begin{document}

\section{China}

\begin{figure}
\caption*{Figure 1: Building metropolitan areas by aggregating smaller units based on
lights at night\label{fig:night-lights-procedure}}
\centering
\subfloat[Inputs]{
\includegraphics[height=0.25\textheight]{../input/nightlight_raster_township_2000_map.png}}
\subfloat[Forming polygons]{
\includegraphics[height=0.25\textheight]{../input/radar_map_townships_2000_30.png}}
\subfloat[Metropolitan areas]{
\includegraphics[height=0.25\textheight]{../input/assignment_map_townships_2000_metros_NTL30.png}}
\begin{center}\begin{minipage}{0.95\textwidth}
{\footnotesize
	\textsc{Notes}:
	This figure illustrates our procedure for combining satellite imagery of lights at night with administrative spatial units to build metropolitan areas.
	These panels depict a portion of the eastern coast of China in 2000.
	The administrative spatial units are townships.
	The polygons in the middle panel are areas of contiguous light brighter than 30.
	Aggregating the townships that intersect these polygons produces the metropolitan areas depicted in the right panel.
	Adjacent townships are often assigned to distinct metropolitan areas.\par
}
\end{minipage}\end{center}
\end{figure}


\begin{table}
\caption*{Table 1: Comparing Chinese township- and county-based metropolitan areas, 2000\label{tab:China-county-vs-township-2000}}
\begin{centering}
\input{../input/msa_compare_China_townships_counties_2000.tex}
\par\end{centering}
\centering{}%
\begin{minipage}[t]{0.9\textwidth}%
{\footnotesize \textsc{Notes}:
The first column reports the number of metropolitan areas with population exceeding 100,000 that are defined by that row's metropolitan scheme.
Each cell in the following six columns reports the correlation coefficient for log population or log land area
between the metropolitan scheme identified in the row and the metropolitan scheme identified
in the column pairs for China in 2000.
We pair metropolitan areas for comparison as described in footnote 8.
\par}
\end{minipage}
\end{table}

\begin{figure}
\caption*{Figure 6: Comparing night-lights--based metropolitan areas to prefecture-level
cities, 2000\label{fig:Comparing-township-prefecturelevel}}
\begin{centering}
\includegraphics[width=0.5\textwidth]{../input/msa_prefecture_baseline_correlation_plot.pdf}
\par\end{centering}
\end{figure}


\begin{table}
\caption*{Table 2: China's city-size distribution with night-lights--based units, 2000
and 2010}
\label{tab:China-zipf-2000-2010}
\begin{center}
\input{../input/msa_compare_zipf_China_townships_2000_2010.tex} \\
\begin{minipage}[t]{0.9\textwidth}%
{\footnotesize \textsc{Notes}:
This table reports the coefficient $\beta$, standard error, and $R^{2}$ from a linear regression of the form
\begin{equation*}
\ln (\text{rank}_i-0.5) = \alpha + \beta \ln \text{population}_i + \epsilon_i
\end{equation*}
where $\text{rank}_i$ is the population rank of metropolitan area $i$
and the standard error is $\sqrt{2/N}\vert\hat{\beta}\vert$ (Gabaix and Ibragimov, 2011).
The sample for each regression is a set of Chinese metropolitan areas in 2000 or 2010
with population greater than 100,000.
Night-lights--based metropolitan areas are defined by aggregating townships in contiguous areas with light intensity exceeding the listed threshold.\par
}
\end{minipage}
\end{center}
\end{table}



\begin{figure}
\caption*{Figure 7: China's city-size distribution with night-lights--based units, 2000
and 2010\label{fig:China-zipf-2000-2010}}
\begin{centering}
\includegraphics[width=0.49\textwidth]{../input/2000_townships_msa30_night_100k.pdf}\includegraphics[width=0.49\textwidth]{../input/2010_townships_msa30_night_100k.pdf}
\par\end{centering}
\centering{}%
\begin{minipage}[t]{0.85\textwidth}%
\textsc{\footnotesize{}Notes}{\footnotesize{}: The sample is Chinese metropolitan
areas with population greater than 100,000. Metropolitan areas defined
by aggregating townships in areas of contiguous night lights with intensity
greater than 30. Left panel depicts 2000; right panel 2010.\par}%
\end{minipage}
\end{figure}

\begin{table} \caption*{Table 6: Population shares for educational categories, 2000} \begin{center}
\input{../input/edu_shares_msa30_night.tex}
\end{center}\end{table}


\begin{table}
\caption*{Table 7: Population elasticities for educational categories, 2000\label{tab:China-edu-population-elasticities}}
\begin{center}
\input{../input/popelast_eduagg_2000_100k_townships_counties.tex}
\begin{minipage}[t]{0.85\textwidth}%
{\footnotesize \textsc{Notes}:
Each column reports OLS estimates of $\beta_{\nu}$ from a regression defined by  equation 1.
Skill fixed effects $\alpha_{\nu}$ are not reported.
Standard errors are clustered by geographic unit.
Each sample contains geographic units with population greater than 100,000.\par}
\end{minipage}
\end{center}
\end{table}


\begin{table}[ph]
\caption*{Table 11: Skill gradient in Chinese metropolitan areas, 2000}
\label{tab:skillgradient}
\begin{center}
\input{../input/distance_gradient_college_2000.tex}
\begin{minipage}{0.96\textwidth}
{\footnotesize \textsc{Notes}:
	The dependent variable is the share of residents who are college graduates in a constituent component,
	which is a township in the upper panel and a county in the lower panel.
	Distance to metro center is measured from the component centroid to the population-weighted average of constituent-component centroids
  	as a share of the greatest such distance in the metropolitan area.
  	The sample is restricted to metropolitan areas containing at least two constituent spatial units.
	Standard errors, clustered by metropolitan area, in parentheses.
	The reported $p$-values test the null hypothesis that the coefficients in the two panels within a column are equal.
	The last line reports the number of clusters used in computing that test statistic.
	See Appendix C.2 for details.\par
}
\end{minipage}
\end{center}
\end{table}

\subsection*{Appendix}

\input{../input/china_townships_area_2000_details.tex}
\input{../input/china_counties_area_2000_details.tex}

\begin{table} \caption*{Table C.1: Pairwise comparisons for educational categories, 2000} \begin{center}
\input{../input/pairwise_edu_2000_4group.tex}
\label{tab:edu_pairwise}
\end{center}\end{table}

\end{document}
