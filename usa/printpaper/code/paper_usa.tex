\documentclass[11pt]{article}
\usepackage{amsmath,amssymb,amsfonts,amsthm}
\usepackage{verbatim}
\usepackage{caption}
\usepackage{graphicx}
\usepackage{natbib}
\usepackage{bibentry}
\usepackage{url}
\usepackage{wrapfig}
\usepackage[margin=1in]{geometry}
\usepackage{framed}
\usepackage{placeins}
\usepackage{bbm}
\usepackage{enumerate}
\usepackage{booktabs}
\usepackage{subfig}   % Supports \subfloat within \figure{} environment
\usepackage{multirow}
\usepackage[setpagesize=false,breaklinks=true]{hyperref}
\hypersetup{
    colorlinks=true,       % false: boxed links; true: colored links
    linkcolor=red,          % color of internal links
    citecolor=black,        % color of links to bibliography
    urlcolor=blue           % color of external links
}

\begin{document}

\section{USA}

\begin{figure}
\caption*{Figure 2: Comparing population and land area across US metropolitan-area definitions, 2010}

\centering{}\includegraphics[width=0.8\textwidth]{../input/US_correlations}
\begin{minipage}{0.8\textwidth}
{\footnotesize
	\textsc{Notes}:
	The left panel depicts correlations of log population and log land area between metropolitan areas defined by contiguous areas of lights at night
	and 377 OMB-defined core-based statistical areas (CBSAs) with population above 100,000 in the 2010 US Census of Population.
	The right panel depicts correlations of log population and log land area between metropolitan areas defined by commuting flows
	and the OMB-defined CBSAs.
	The horizontal axes vary the thresholds for light intensity (left panel) and commuting flows (right panel) used to define metropolitan areas in our procedure and the Duranton (2015) procedure, respectively.
	Footnote 8 describes how we pair CBSAs with comparison counterparts.\par
}
\end{minipage}
\end{figure}


\end{document}
